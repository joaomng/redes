\documentclass[a4paper,12pt]{article}
\usepackage[style=authoryear,sorting=ynt, maxbibnames=2]{biblatex}
\usepackage[unicode, draft=false]{hyperref}
\usepackage[scale=0.9]{geometry}
\usepackage{xcolor}

\title{Teleprocessamento e Redes - Relatório do trabalho final}
\author{
  Leonardo Ribeiro Santiago (120036072) \\
  João Matheus Nascimento Gonçálves (117209640) \\
  Esteves Emmanuel Melo Ferreira (120023786) }

\date{}

\begin{document}

\maketitle

\section{Introdução}

Neste relatório iremos responder as perguntas relacionadas à parte 2 do trabalho final. O código está disponível no seguinte repositório do github: {\color{blue} \href{https://github.com/o-santi/redes}{github.com/o-santi/redes}}.

Para reproduzir os resultados, deve-se instanciar uma máquina virtual Ubuntu usando Vagrant, assim como descrito em {\color{blue} \href{https://github.com/kaichengyan/mininet-vagrant}{github.com/kaichengyan/mininet-vagrant}}. Uma vez dentro da VM, clonamos o repositório git para uma pasta interna, e rodamos o arquivo \texttt{run.sh}.
\begin{verbatim}
  git clone https://github.com/o-santi/redes.git ~/redes
  cd ~/redes
  chmod +x ./bufferbloat/run.sh
  sudo ./bufferbloat/run.sh
\end{verbatim}

Isto irá rodar os dois casos de teste (\texttt{max\_queue=20} e \texttt{max\_queue=100}) e gerar os gráficos citados neste relatório.

\section{Parte 2}

\subsection{Qual é o tempo médio de busca da página da web e seu desvio padrão quando q=20 e q=100?}

\subsection{Por que você vê uma diferença nos tempos de busca de páginas da Web com buffers de roteador curtos e grandes?}

\subsection{Bufferbloat pode ocorrer em outros lugares, como sua placa de interface de rede (NIC). Verifique a saída de ifconfig eth0 de sua VM mininet. Qual é o comprimento (máximo) da fila de transmissão na interface de rede relatada pelo ifconfig? Para esse tamanho de fila, se você assumir que a fila é “drenada” a 100 Mb/s, qual é o tempo máximo que um pacote pode esperar na fila antes de sair da NIC?
}

\subsection{Como o RTT relatado pelo ping varia com o tamanho da fila? Descreva a relação entre os dois.}

\subsection{Identifique e descreva duas maneiras de mitigar o problema de bufferbloat.}

%% \section{Parte 3}


\end{document}
